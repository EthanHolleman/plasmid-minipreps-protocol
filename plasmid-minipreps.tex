\documentclass[12pt]{article}
\usepackage[utf8]{inputenc}	% Para caracteres en español
\usepackage{amsmath,amsthm,amsfonts,amssymb,amscd}
\usepackage{multirow,booktabs}
\usepackage[table]{xcolor}
\usepackage{fullpage}
\usepackage{lastpage}
\usepackage{enumitem}
\usepackage{fancyhdr}
\usepackage{mathrsfs}
\usepackage{wrapfig}
\usepackage{graphicx}
\usepackage{caption}
\usepackage{subcaption}
\usepackage{setspace}
\usepackage{calc}
\usepackage{multicol}
\usepackage{cancel}
\usepackage[T1]{fontenc} 
%\usepackage[retainorgcmds]{IEEEtrantools}
\usepackage[margin=3cm]{geometry}
\usepackage{amsmath}
\newlength{\tabcont}
\setlength{\parindent}{0.0in}
\setlength{\parskip}{0.05in}
\usepackage{empheq}
\usepackage{framed}
\usepackage[most]{tcolorbox}
\usepackage{xcolor}
\colorlet{shadecolor}{orange!15}
\parindent 0in
\parskip 12pt
\usepackage[T1]{fontenc}

 \renewcommand{\familydefault}{\sfdefault}
\geometry{margin=1in, headsep=0.25in}
\theoremstyle{definition}


\usepackage{graphicx}
\usepackage{hyperref}
\hypersetup{
	colorlinks=true,
	linkcolor=blue,
	filecolor=magenta,      
	urlcolor=cyan,
}

\begin{document}
\title{
	\textbf{Plasmid minipreps for sequencing}\\
	\large Adapted from \emph{FLC Protocols Plasmid minipreps 02/09/2004}}

\author{Ethan Holleman}
\maketitle

\section*{Reagents}

Below are reagents you will need to prepare before starting the protocol and how to prepare them.

\subsection*{Solution I: Re-suspension buffer}
\label{sec:sol-1}

50 mM Tris-Cl, pH 8.0, 10mM EDTA, 100 $\mu$g/mL RNaseA.

\begin{enumerate}
	\item Dissolve 6.06 g Tris base, 3.72g EDTA-2H$_2$0 in 800mL npH$_2$0. Adjust the pH to 8.0 with HCL. 
	\item Adjust the volume to 1 liter with npH$_2$O. Add 100 mg RNaseA per liter of Solution I. Store at -4 $^{\circ}$C after adding RnaseA.
\end{enumerate}

\subsection*{Solution II: Lysis buffer}
\label{sec:sol-2}

200mM NaOH, 1\% SDS.

\begin{enumerate}
	\item Dissolve 8.09 g of NaOH pellets in 950mL npH$_2$O. 
	\item  Add 50 mL 20\% SDS solution. If making SDS solution stock be sure to add SDS to water slowly. Store at room temperature.
\end{enumerate}

\subsection*{Solution III: Neutralization Buffer}
\label{sec:sol-3}

3.0 M potassium acetate, pH 5.5 .

\begin{enumerate}
	\item Dissolve 294.5 g potassium acetate in 500 mL npH$_2$O.
	\item Adjust the pH to 5.5 with glacial acetic acid (about 110 mL). Store at -4$^{\circ}$C or room temperature
\end{enumerate}

\subsection*{10 M LiCl}
\label{sec:licl}

\begin{enumerate}
	\item Dissolve 42.4 g of LiCl in 90 ml npH$_2$O.
	\item Adjust volume to 100 ml. Store at room temperature.
\end{enumerate}

\subsection*{10 mM Tris HCL pH 7.5}
\label{sec:tris}

If stock is not on hand make a 1 M Tris-HCL solution, otherwise skip to step 3.

\begin{enumerate}
	\item Dissolve 121.1 g Tris base in 800 ml npH$_2$O.
	\item Adjust the pH to 7.5 using HCL.
	\item Add 5 ml of 1 M Tris-HCL pH 7.5 to 495 ml npH$_2$O.
\end{enumerate}

If making RnaseA solution, add RnaseA stock to a final concentration of 20 $\mu$g / ml. 


\subsection*{Other items}

\begin{itemize}
	\item 100\% ethanol.
	\item 70\% ethanol.
	\item Lots of clean (autoclaved) 2 ml tubes.
	\item 3-5 hours of time to complete the protocol.
\end{itemize}

\pagebreak

\section*{Protocol}


\begin{enumerate}
	\item Begin with a minimum of 2 and a maximum of 6 ml of overnight bacterial culture (if ampicillin is used as the selective medium use 100 $\mu$/ml). If volume of culture is greater than 2 ml proceed to step 2, otherwise skip to step 4.
	
	\item Transfer culture to a 12 ml tube. Spin down bacteria at 4500 rpm at 23 $^{\circ}$C for 10 minutes using the large bench-top centrifuge and the swing bucket rotor \footnote{Plastic 12 ml tubes will shatter if spun at too high a speed with too great a volume. If you have doubts about your tubes or have not used them in the centrifuge before test them by spinning a few with 11 $\mu$l H$_{2}$O at 4200 rpm.}.
	
	\item Transfer culture to a 2ml centrifuge tube and spin down bacteria at 6,000 rpm at 23 $^{\circ}$C for 6 minutes.
	
	\item Add 200 $\mu$l \hyperref[sec:sol-1]{solution I} \footnote{I highly recommend using the serial pippetter for adding all solutions at this point. It will save you hours over traditional pipetting. If you do not know what this is or where to find it ask someone.}. 
	Re-suspend pellets by vortexing vigorously. If using > 2 ml of culture (see step 2) transfer
	re-suspended bacteria to a 2 ml tube. Incubate samples at room temperature for 10 minutes. While you are waiting cool down the 2 ml bench-top centrifuge to 4  $^{\circ}$C.
	
	\item Add 400 $\mu$l of \hyperref[sec:sol-2]{solution II}. Mix the samples by gently
	inverting 6 times. \textbf{Do not vortex}. Place the tubes on ice for exactly 5 minutes.
	
	\item Add 300 $\mu$l of cold \hyperref[sec:sol-3]{solution III}. Incubate samples on ice for 5 minutes.
	
	\item Centrifuge for 5 minutes at max speed (15,000 rpm) at 4  $^{\circ}$C.
	
	\item Transfer the supernatant to a clean 2 ml tube. Add 1 volume (900 $\mu$l) of isopropanol. Mix the samples by inverting several times.
	
	\item Centrifuge samples for 15 minutes at max speed at 4  $^{\circ}$C.
	
	\item Discard supernatent by decanting or with vacuum trap. Wash pellet with 500 $\mu$l of 70\% ethanol. \footnote{Generally I decant the isopropanol by placing samples in the front row of a tube rack, facing them so the front of the tube faces away from the rows of the rack and using this configuration to decant all the samples in the rack at once. It helps to find a tube rack that is older and slightly deformed as these will hold onto your tubes better.}
	
	\item Add 200 $\mu$l \hyperref[sec:tris]{10mM Tris HCL RnaseA} solution to each sample. Incubate samples at 37 $^{\circ}$C for 10 minutes.
	
	\item Add 200 $\mu$l of \hyperref[sec:licl]{10M LiCl}. Mix solutions by inverting 6 times.
	
	\item Incubate samples for $ \geq$ 20 minutes at -20 $^{\circ}$C.
	
	\item  Centrifuge samples at max speed for 10 minutes at 4  $^{\circ}$C. Transfer the \textbf{supernatent} to a clean 2 ml tube.
	
	\item Add 2.2 volumes of ice cold 100\% ethanol. Mix samples by inverting several times.
	
	\item Centrifuge samples at max speed for 10 minutes at 4  $^{\circ}$C. Remove supernatent with a vacuum trap. \footnote{Remember to use a pipette tip over the end of the vacuum tube.}
	
	\item Wash the pellet up to two times with 500 $\mu$l of 70\% ethanol \footnote{The 2/9/04 protocol recommends washing twice. I haven't noticed a difference between single and double wash samples but if you are being extra careful it doesn't hurt to wash twice.}. Remove ethanol using a vacuum trap. Washing can sometimes dislodge the pellet and it can be helpful to spin samples at max speed for 2 minutes prior to removing the wash to secure the pellet.
	
	\item Dry the pellet for 10-15 minutes and dissolve in 40 $\mu$l \hyperref[sec:tris]{10mM Tris HCL pH 7.5}. 
	
	\item Check sample concentration using the nanodrop. If concentration is too high dilute sample in 40 $\mu$l steps of \hyperref[sec:tris]{10mM Tris HCL pH 7.5} until you reach a reasonable concentration (200-400 ng/$u$l).
	
	\item Run between 200 and 600 ng per lane of each sample on an agarose gel and image.
	

\end{enumerate}


\end{document}

